
\section {Convolution}

\subsection{Constructing hemodynamic response function}

We can use the gamma function to construct a continuous function that is close to 
the hemodynamic response we observe for different events in the brain. In our case,
we have 4 different events (conditions): task = \{0,1\}, parametric gain, parametric loss, distance 
from indifference. However, we cannot really convolve the canonical hemodynamic response function with the neural prediction from text files with condition information because:\newline
\newline
\indent 1. The events are given at the different onsets in the experiment.\newline
\indent 2. The hrf lasts for 30 seconds when the duration for each event is 3 seconds.\newline
\newline
Therefore, we tried the normal way to convolve the canonical hrf and the other way to convolve with hrf at the higher time resolution, and compared the resulting MRSS from the general linear regression analysis using the predictor from the normal convolution, and it from the other convolution.\newline


\begin{itemize}
\item Convolve with the canonical hemodynamic response function:\newline
\indent There are 4 condition files.\newline
\indent use the sum of two gamma distribution probability density functions\newline
\indent sample from the function, to get the estimates at the times of our TRs (2)\newline
\indent use this to convolve our neural (on-off) prediction\newline
\indent We applied this method to all 4 condition files samely and built the design matrix for the linear regression analysis. \newline


\item Convolve with the high resolution neural time course:\newline
\indent There are 4 condition files.\newline
\indent make a neural and hemodynamic regressor at a finer time resolution than the TRs\newline
\indent create a new neural prediction time-course where one element corresponds to 1 / 100 of a TR\newline
\indent sample at the TR onset times.
\indent convolve the sampled HRF with the high resolution neural time course\newline



--$>$ If we compare the MRSS from two linear regressions for three subjects (1,2,3) using convolution predictors from two different methods in the below table, we see the MRSS from linear regression using the latter method has slightly lower values compared to the former method. This makes sense because, using the latter method, we could more elaborately preprocess the data.
\end {itemize}
