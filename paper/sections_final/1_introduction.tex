\section{Introduction}
\noindent
The study \textit{Neural Basis of Loss Aversion in Decision-Making Under Risk} 
\cite{tom2007neural} focuses on decision-making process, especially on the 
correlation between the neural activity and the reluctance to lose. 16 people
were presented 255 gambling situations with a 50\% of success. Each situation
was associated with a potential gain and loss that were randomly selected.
The gains were ranging from \$10 to \$40 while the losses from \$5 to \$20.
The participants were asked to assess their level of willingness to accept or 
reject the gamble using a 4-point likert scale [1: strongly accept,
2: weakly accept, 3: weakly reject, 4:strongly reject]. The response time was
also recorded for each case.
The imaging data were collected using the fMRI method. They were processed 
and analyzed in order to identify the regions of the brain activated by the
decision making process. This study also investigated the relationship between 
the brain activity and the behavior of the subjects towards the gambling
situations using a whole-brain robust regression analysis.
