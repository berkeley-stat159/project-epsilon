\section{Problems and Issues}
\noindent
When approaching the project, we have faced many problems. They can be divided 
into three main categories.
\subsection{understand the data}
\noindent
The first main problem that we have encountered is about understanding the 
data. Although we have a large dataset, there is no specific file that 
introduces the data structure or explain the meaning of the variables. Due to 
the lack of documentation of these relative information, we have spent much 
time reading through the paper and searching the website. Even so, there are 
still some problems unsolved: for example, we had no ideas about the variable 
called "PTval" in the behavior data and did not use it for our analysis. 
Meanwhile, since the data we are analyzing is fMRI data and we are all 
unfamiliar with its format and processing procedure, therefore, in the 
beginning, we found that the technical field studies are difficult to 
understand. 
\noindent
The insufficient description of the analysis methods also made us hard to 
reproduce the work. For example,  there are no descriptions about the QA 
section. When we are trying to reproduce the plots or calculate certain 
statistics, we can not get the exact results: the plots have the same shapes 
and patterns but the values on the y-axis are more than two times smaller than 
those in the QA report. As a result, we doubted that they might perform some 
transformations on the data set but did not mention them in the paper. We also 
attempted to do a generalized logistic regression for the behavioral data by 
including gender and age of each subjects, but there was no specific guidance 
on how to perform this.

\subsection{Coding}
\noindent
The second main problem that we have faced is about the coding. In order to 
assure the reproducibility of our analysis, we need to do everything from the 
terminal, instead of simple point and click. We write a lot of functions, 
scripts and tests to download, unzip, load the files and travel through 
different folders.
\noindent
We also use Git to do version control. However, it might cause some conflicts 
or misunderstandings when we just merge the pull requests. In order to avoid 
these situations, we keep track of others’ code, try to add more descriptions 
to our code and review others' pull request more carefully.

\subsection{Data Analysis}
\noindent
The hardest problems come from the data analysis. Our image data and behavior 
data for each run and each individual are in different folders, so sometimes 
it's hard to work on data across subjects and runs.
Meanwhile, the length of behavior data, which is 239, does not match the length 
of image data, which is only 86. Therefore, when we tried to combine them 
together and check if there are some linear relationships between them, we 
didn't how how to deal with the difference. If we fill in NA, we will lose much 
information; if we fill in 0, it is meaningless to run a linear regression that 
have many zeros in it.
\noindent
We also have many questions regards to the convolution. In the lecture or our 
homework, the convolution matrix is given to us, but now, we need to create a 
convolution matrix by ourselves. We are also unable to determine the parameters 
of the gamma distribution for the HRF. Most of online papers are using gamma 
quadratic (2 gamma functions + intercept term) and the common practice is to 
use 6 and 12 and an intercept of -0.35; however, the results are not reasonable 
for our project. We will continue working on it, try different parameters and 
find more online sources.

